
The availability of high-density low-cost silicon probes \citep{jun2017fully} has led to a step change in the scale \citep{steinmetz2019distributed} and ambition \citep{koch2022next} of modern neuroscience. This significant advance in hardware has not been fully matched by a similar improvement in the software interfaces used for performing experiments. Although some aspects of performing electrophysiology experiments are now possible through intuitive open-source interfaces, including data acquisition \citep{siegle2017open}; planning experiments, performing surgical targeting, and assessing probe position in real time continues to require significant expertise on the part of researchers. Experimental design should not require researchers to have expert anatomical knowledge of the entire brain, instead, experimental planning tools should provide users with easy-to-use and intuitive interfaces that guide them in designing and executing optimal probe insertion plans for their experimental goals. 